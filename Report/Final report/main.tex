\documentclass[10pt,twocolumn,letterpaper]{article}

\usepackage{cvpr}
\usepackage{times}
\usepackage{epsfig}
\usepackage{graphicx}
\usepackage{amsmath}
\usepackage{amssymb}

% Include other packages here, before hyperref.

% If you comment hyperref and then uncomment it, you should delete
% egpaper.aux before re-running latex.  (Or just hit 'q' on the first latex
% run, let it finish, and you should be clear).
\usepackage[breaklinks=true,bookmarks=false]{hyperref}

\cvprfinalcopy % *** Uncomment this line for the final submission

\def\cvprPaperID{****} % *** Enter the CVPR Paper ID here
\def\httilde{\mbox{\tt\raisebox{-.5ex}{\symbol{126}}}}

% Pages are numbered in submission mode, and unnumbered in camera-ready
%\ifcvprfinal\pagestyle{empty}\fi
\setcounter{page}{1}
\begin{document}

%%%%%%%%% TITLE
\title{Object recognition \& Computer vision \\ Final report - MVA 2015 \\ \textit{Functional Maps: A Flexible Representation of Maps Between Shapes}}

\author{PAUMIER Nicolas\\
ENS Cachan\\
{\tt\small paumiern.ensimag@gmail.com }
% For a paper whose authors are all at the same institution,
% omit the following lines up until the closing ``}''.
% Additional authors and addresses can be added with ``\and'',
% just like the second author.
% To save space, use either the email address or home page, not both
\and
THIS Alexandre\\
ENS Cachan\\
{\tt\small alexandre.this@gmail.com}
}

\maketitle
%\thispagestyle{empty}

%%%%%%%%% ABSTRACT
\begin{abstract}

\end{abstract}

%%%%%%%%% BODY TEXT
\section{Abstract}
%TODO at the end

\section{Introduction}
%Statement of the problem, quick recap of previous work

\section{Method}
%Description de la fonctional map

\section{Implementation}
%Description de comment compute les constraints

\section{Results}
%Tests réalisés

\section{Discussions}
%Analyse des résultats


%-------

{\small
\bibliographystyle{ieee}
\bibliography{egbib}
}

\end{document}
